\documentclass[10pt, oneside]{report} 
\usepackage{amsmath, amsthm, amssymb, calrsfs, wasysym, verbatim, bbm, color, graphics, geometry}

\usepackage{physics}

\usepackage{trimclip}
\newcommand{\halfapprox}{\mathchoice{%
\clipbox{0em 0em 0em 0.22em}{$\displaystyle\approx$}}{%
\clipbox{0em 0em 0em 0.22em}{$\textstyle\approx$}}{%
\clipbox{0em 0em 0em 0.18em}{$\scriptstyle\approx$}}{%
\clipbox{0em 0em 0em 0.18em}{$\scriptscriptstyle\approx$}}}

\usepackage{titlesec}
\setcounter{secnumdepth}{3}

\usepackage{hyperref}

\geometry{tmargin=.75in, bmargin=.75in, lmargin=.75in, rmargin = .75in}  

\newcommand{\O}{\mathbb{O}}
\newcommand{\R}{\mathbb{R}}
\newcommand{\C}{\mathbb{C}}
\newcommand{\Z}{\mathbb{Z}}
\newcommand{\N}{\mathbb{N}}
\newcommand{\Q}{\mathbb{Q}}
\newcommand{\E}{\mathbb{E}}
\newcommand{\Cdot}{\boldsymbol{\cdot}}

\newtheorem{thm}{Theorem}
\newtheorem{defn}{Definition}
\newtheorem{conv}{Convention}
\newtheorem{rem}{Remark}
\newtheorem{lem}{Lemma}
\newtheorem{cor}{Corollary}
\newtheorem{algo}{Algorithm}
\newtheorem{clm}{Claim}
\newtheorem{pse}{Pseudocode}


\title{Physics Informed Neural Networks}
\author{Nicholas Leland}
\date{November 2024}

\begin{document}

\maketitle
\tableofcontents

\vspace{.25in}

\chapter{Partial Differential Equations}
What is a Physics-Informed neural network? These are also known as \textbf{Theory-Trained Neural Networks}, our goal is to \textit{learn} the physical laws that govern a dataset, often laws that are typically expressed utilizing \textbf{Partial Differental Equations}.  
In essence, Partial differential equations are used to construct models of the most basic theories underlying physics and engineering.  



\section{sample}
test
$\pdv{Q}{t}$

\subsection{sample}
test

\chapter{Physic Informed Neural Networks, Introduction}
test

\end{document}


